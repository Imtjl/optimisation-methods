\documentclass[a4paper]{article}
\usepackage[14pt]{extsizes}
\usepackage[russian]{babel}
\usepackage[left=20mm, top=15mm, right=15mm, bottom=15mm]{geometry}
\usepackage{listings}
\usepackage[utf8]{inputenc}
\usepackage[T1]{fontenc}
\usepackage{listingsutf8}
\usepackage{xcolor}
\usepackage{graphicx}
\usepackage{amsmath}
\usepackage{amssymb}
\usepackage{hyperref}
% Дополнительная настройка для языка Rust
\lstdefinelanguage{Rust}{
    morekeywords={true, false, match, let, mut, ref, enum, fn, impl, pub, self, Self, use, super, trait, where, as, struct},
    morecomment=[l]{//},
    morecomment=[s]{/*}{*/},
    morestring=[b]",
    morestring=[s]{r#"}{"#},
    morestring=[s]{r##"}{"##},
    morestring=[s]{r###"}{"###},
}

% Настройка вида листинга
\lstset{
    language=Rust,
    basicstyle=\ttfamily,
    keywordstyle=\color{blue},
    stringstyle=\color{red},
    commentstyle=\color{green},
    numbers=left,
    numberstyle=\tiny,
    stepnumber=1,
    numbersep=5pt,
    backgroundcolor=\color{white},
    showspaces=false,
    showstringspaces=false,
    showtabs=false,
    frame=single,
    rulecolor=\color{black},
    tabsize=4,
    captionpos=b,
    breaklines=true,
    breakatwhitespace=false,
    escapeinside={\%*}{*)},
    morekeywords={match, let, mut, ref, enum, fn, impl, pub, self, Self, use, super, trait, where, as, struct}, % добавляем ключевые слова Rust
}


% -----------------------------------------------------

\begin{document} 

% Title page
\begin{center}
\hfill \break
\large{Университет ИТМО}\\
\large{Факультет программной инженерии и компьютерной техники}\\ 
\large{Направление подготовки: 09.03.01 - Информатика и вычислительная техника, Компьютерные системы и технологии}\\ 
\hfill\break
\hfill \break
\hfill \break
\hfill \break
\hfill \break
\hfill \break
\hfill \break
\hfill \break
\large{Лабораторная работа №1\\по Методам оптимизации\\}
\hfill \break
\hfill \break
\hfill \break
\hfill \break
\hfill \break
\end{center}
\hfill \break
\normalsize{ 
\begin{flushright}
Выполнил:
\par
Дворкин Борис Александрович
\par
Преподаватель:
\par
Селина Елена Георгиевна
\end{flushright}
}\\
\hfill \break
\hfill \break
\hfill \break
\hfill \break
\begin{center} г. Санкт-Петербург 
\par
2024 г. 
\end{center}
\thispagestyle{empty}
\thispagestyle{empty}

% Description
\section{Программная реализация вычислительных методов оптимизации}
Алгоритмы оптимизации реализованы на языке Rust. В следующих подразделах подробно описана реализация каждого метода.


% Code
\subsection{Программная реализация}

\begin{lstlisting}
import random
import numpy as np

MUTATION_RATE = 0.01

def calculate_path_sum(path, dist_matrix):
    return (
        sum(dist_matrix[path[i]][path[i + 1]] for i in range(len(path) - 1))
        + dist_matrix[path[-1]][path[0]]
    )

def crossover(parent1, parent2, points):
    child = (
        parent1[:points[0]]
        + parent2[points[0] : points[1]]
        + parent1[points[1]:]
    )
    return child

def mutate(individual):
    if random.random() < MUTATION_RATE:
        idx1, idx2 = random.sample(range(len(individual)), 2)
        individual[idx1], individual[idx2] = individual[idx2], individual[idx1]
        return True
    return False


def create_population(size, num_cities):
    return [random.sample(range(num_cities), num_cities) for _ in range(size)]

def select_parents(population, fitnesses):
    return [
        population[i]
        for i in np.random.choice(
            len(population), size=2, p=fitnesses / sum(fitnesses), replace=False
        )
    ]

def print_population_info(population, fitnesses, generation):
    print(f"Generation {generation+1}:")
    print("Population:", population)
    print("Distances:", fitnesses)

def genetic_algorithm(num_cities, dist_matrix, pop_size, num_generations):
    population = create_population(pop_size, num_cities)

    for generation in range(num_generations):
        fitnesses = np.array(
            [calculate_path_sum(route, dist_matrix) for route in population]
        )
        print_population_info(population, fitnesses, generation)

        new_population = []
        parent_pairs = [
            select_parents(population, fitnesses) for _ in range(pop_size // 2)
        ]
        for pair_idx, (parent1, parent2) in enumerate(parent_pairs):
            print(f"\nPair {pair_idx + 1}: {[population.index(parent1), population.index(parent2)]}")
            points = sorted(random.sample(range(1, num_cities), 2))
            print(
                "Parent 1: "
                + " ".join(map(str, parent1[:points[0]]))
                + " | "
                + " ".join(map(str, parent1[points[0] : points[1]]))
                + " | "
                + " ".join(map(str, parent1[points[1] :]))
            )
            print(
                "Parent 2: "
                + " ".join(map(str, parent2[:points[0]]))
                + " | "
                + " ".join(map(str, parent2[points[0] : points[1]]))
                + " | "
                + " ".join(map(str, parent2[points[1] :]))
            )
            child1, child2 = crossover(parent1, parent2, points), crossover(
                parent2, parent1, points
            )
            print("Child 1: " + " ".join(map(str, child1)))
            print("Child 2: " + " ".join(map(str, child2)))
            if mutate(child1):
                print("Child 1 MUTATED: " + " ".join(map(str, child1)))
            if mutate(child2):
                print("Child 2 MUTATED: " + " ".join(map(str, child2)))
            new_population.extend([child1, child2])

        population = sorted(
            population + new_population,
            key=lambda x: calculate_path_sum(x, dist_matrix),
        )[:pop_size]
        print()
        print("Enlarged population:", population)
        fitnesses = [calculate_path_sum(route, dist_matrix) for route in population]
        print("Distances:", fitnesses)
        population = population[:pop_size]
        print()

    best_route = min(population, key=lambda x: calculate_path_sum(x, dist_matrix))
    return best_route, calculate_path_sum(best_route, dist_matrix)

if __name__ == "__main__":
    num_cities = int(input("Enter the number of cities: "))
    print("Enter the distance matrix:")
    dist_matrix = [list(map(int, input().split())) for _ in range(num_cities)]
    pop_size = int(input("Enter the population size: "))
    num_generations = int(input("Enter the number of generations: "))

    best_route, best_distance = genetic_algorithm(
        num_cities, dist_matrix, pop_size, num_generations
    )
    print(f"\nBest route: {best_route}, Distance: {best_distance}")
\end{lstlisting}

\subsection{Пользовательский ввод}

\begin{verbatim}
Enter number of cities: 5
Enter matrix: 
0 4 5 3 8
4 0 7 6 8
5 7 0 7 9
3 6 7 0 9
8 8 9 9 0
Enter population size: 4
Enter number of generations: 2
\end{verbatim}

\subsection{Вывод программы}

\begin{verbatim}
GENERATION 1
Population:
[[4, 2, 0, 3, 1], [2, 1, 3, 0, 4], [1, 4, 0, 2, 3], [2, 1, 0, 4, 3]]
Lengths:
[31, 33, 34, 35]

Pair 1: [0, 1]
Parent 1: 4 2 | 0 3 | 1
Parent 2: 2 1 | 3 0 | 4
Child 1: 1 4 3 0 2
Child 2: 4 2 0 3 1

Pair 2: [0, 1]
Parent 1: 4 | 2 0 3 1 | 
Parent 2: 2 | 1 3 0 4 | 
Child 1: 2 1 3 0 4
Child 2: 4 2 0 3 1

Enlarged population:
[[4, 2, 0, 3, 1],
 [4, 2, 0, 3, 1],
 [4, 2, 0, 3, 1],
 [1, 4, 3, 0, 2],
 [2, 1, 3, 0, 4],
 [2, 1, 3, 0, 4],
 [1, 4, 0, 2, 3],
 [2, 1, 0, 4, 3]]
Lengths:
[31, 31, 31, 32, 33, 33, 34, 35]

GENERATION 2
Population:
[[4, 2, 0, 3, 1], [4, 2, 0, 3, 1], [4, 2, 0, 3, 1], [1, 4, 3, 0, 2]]
Lengths:
[31, 31, 31, 32]

Pair 1: [0, 2]
Parent 1: 4 2 0 | 3 1 | 
Parent 2: 4 2 0 | 3 1 | 
Child 1: 4 2 0 3 1
Child 2: 4 2 0 3 1

Pair 2: [1, 3]
Parent 1:  | 4 2 0 3 | 1
Parent 2:  | 1 4 3 0 | 2
Child 1: 1 4 3 0 2
Child 2: 4 2 0 3 1

Enlarged population:
[[4, 2, 0, 3, 1],
 [4, 2, 0, 3, 1],
 [4, 2, 0, 3, 1],
 [4, 2, 0, 3, 1],
 [4, 2, 0, 3, 1],
 [4, 2, 0, 3, 1],
 [1, 4, 3, 0, 2],
 [1, 4, 3, 0, 2]]
Lengths:
[31, 31, 31, 31, 31, 31, 32, 32]

GENERATION 3
Population:
[[4, 2, 0, 3, 1], [4, 2, 0, 3, 1], [4, 2, 0, 3, 1], [4, 2, 0, 3, 1]]
Lengths:
[31, 31, 31, 31]

Pair 1: [2, 1]
Parent 1: 4 2 | 0 3 | 1
Parent 2: 4 2 | 0 3 | 1
Child 1: 1 4 0 3 2
Child 2: 1 4 0 3 2

Pair 2: [1, 0]
Parent 1: 4 | 2 0 3 1 | 
Parent 2: 4 | 2 0 3 1 | 
Child 1: 4 2 0 3 1
Child 2: 4 2 0 3 1

Enlarged population:
[[4, 2, 0, 3, 1],
 [4, 2, 0, 3, 1],
 [4, 2, 0, 3, 1],
 [4, 2, 0, 3, 1],
 [4, 2, 0, 3, 1],
 [4, 2, 0, 3, 1],
 [1, 4, 0, 3, 2],
 [1, 4, 0, 3, 2]]
Lengths:
[31, 31, 31, 31, 31, 31, 33, 33]

Result after 3 generations is [4, 2, 0, 3, 1] with length 31
\end{verbatim}

% Forth section - Examples of code usage
% \input{examples}

% Fith section - Conclusion
\section{Вывод программы}
Метод половинного деления: 0.734375 \\
Метод золотого сечения: 0.7360679774997897 \\
Метод хорд: 0.736298997613654 \\
Метод Ньютона: 0.7552224171056364 \\

\end{document}