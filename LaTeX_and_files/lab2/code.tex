\subsection{Демонстрация работы}
\begin{lstlisting}
mod methods;

use methods::{Optimizer, Function};

struct MyFunction;
impl Function for MyFunction {
    fn f(&self, x: f64) -> f64 {
        0.5 * x.powi(2) - x.sin()
    }

    fn f_deriv(&self, x: f64) -> f64 {
        x - x.cos()
    }

    fn f_deriv2(&self, x: f64) -> f64 {
        1.0 + x.sin()
    }
}

fn main() {
    let func = MyFunction;
    let opt = Optimizer::new(0.0, 1.0, 0.03, &func);
    println!("Bisection {}", opt.bisection());
    println!("Golden section: {}", opt.golden_section());
    println!("Chord: {}", opt.chord());
    println!("Newton: {}", opt.newton());
}
\end{lstlisting}

\subsection{Общие определения}
\begin{lstlisting}
pub mod bisection;
pub mod chord;
pub mod golden_section;
pub mod newton;

pub trait Function {
    fn f(&self, x: f64) -> f64;
    fn f_deriv(&self, x: f64) -> f64;
    fn f_deriv2(&self, x: f64) -> f64;
}

pub struct Optimizer<'a> {
    a: f64,
    b: f64,
    eps: f64,
    func: &'a dyn Function,
}

impl<'a> Optimizer<'a> {
    pub fn new(a: f64, b: f64, eps: f64, func: &'a dyn Function) -> Self {
        Optimizer { a, b, eps, func }
    }

    pub fn bisection(&self) -> f64 {
        bisection::bisection(self.a, self.b, self.eps, self.func)
    }

    pub fn golden_section(&self) -> f64 {
        golden_section::golden_section(self.a, self.b, self.eps, self.func)
    }

    pub fn chord(&self) -> f64 {
        chord::chord(self.a, self.b, self.eps, self.func)
    }

    pub fn newton(&self) -> f64 {
        newton::newton(self.a, self.b, self.eps, self.func)
    }
}
\end{lstlisting}

\subsection{Метод половинного деления}
\begin{lstlisting}
use super::Function;

pub fn bisection(mut a: f64, mut b: f64, eps: f64, func: &dyn Function) -> f64 {
    while (b - a) > 2.0 * eps {
        let x = (a + b) / 2.0;
        if func.f(x - (eps / 2.0)) > func.f(x + (eps / 2.0)) {
            a = x;
        } else {
            b = x;
        }
    }
    (a + b) / 2.0
}
\end{lstlisting}

\subsection{Метод золотого сечения}
\begin{lstlisting}
use super::Function;

pub fn golden_section(mut a: f64, mut b: f64, eps: f64, func: &dyn Function) -> f64 {
    let phi = (1.0 + 5.0_f64.sqrt()) / 2.0;
    let mut x1 = a + (2.0 - phi) * (b - a);
    let mut x2 = a + (phi - 1.0) * (b - a);

    while (b - a) > 2.0 * eps {
        if func.f(x1) < func.f(x2) {
            b = x2;
            x2 = x1;
            x1 = a + (2.0 - phi) * (b - a);
        } else {
            a = x1;
            x1 = x2;
            x2 = a + (phi - 1.0) * (b - a);
        }
    }
    (a + b) / 2.0
}
\end{lstlisting}

\subsection{Метод хорд}
\begin{lstlisting}
use super::Function;

pub fn chord(mut a: f64, mut b: f64, eps: f64, func: &dyn Function) -> f64 {
    loop {
        let f_a = func.f_deriv(a);
        let f_b = func.f_deriv(b);
        let x = a - f_a * (a - b) / (f_a - f_b);

        let f_x = func.f_deriv(x);
        if f_x.abs() <= eps {
            return x;
        }

        if f_x > 0.0 {
            b = x;
        } else {
            a = x;
        }
    }
}
\end{lstlisting}

\subsection{Метод Ньютона}
\begin{lstlisting}
use super::Function;

pub fn newton(a: f64, b: f64, eps: f64, func: &dyn Function) -> f64 {
    let mut x = (a + b) / 2.0;
    while func.f_deriv(x).abs() > eps {
        x = x - func.f_deriv(x) / func.f_deriv2(x);
    }
    x
}
\end{lstlisting}